\documentclass[11pt]{article}
\usepackage[margin=1in]{geometry}
\usepackage{amsmath,amsfonts,amssymb,amsthm,enumerate}
\usepackage[]{graphicx}
\usepackage{color,subfigure}
\usepackage{multicol}
\usepackage{float}
\usepackage[all]{xypic}
\usepackage[colorlinks=true,citecolor=cyan,linkcolor=magenta]{hyperref}
\usepackage{colonequals}

\usepackage{fancyhdr, lastpage}
\pagestyle{fancy}
\fancyfoot[C]{{\thepage} of \pageref{LastPage}}


\title{}
\date{\vspace{-0.5in}}

\makeatletter
\g@addto@macro\@floatboxreset\centering
\makeatother

\theoremstyle{definition}
\newtheorem{problem}{Problem}


\title{Homework 1}

\begin{document}

\thispagestyle{fancy}
\pagestyle{fancy}
\rhead{UCR $\mid$ Elo\'isa Grifo}
\lhead{Commutative Algebra Winter 2021}


\begin{center}
	{\LARGE Problem Set 2}
\end{center}



\begin{problem}
Given two ideals $I$ and $J$, the \emph{product} of $I$ and $J$ is the ideal
$$IJ \colonequals \left( fg \mid f \in I \textrm{ and } g \in J \right) .$$
\begin{enumerate}[a)]
	\item Show that $IJ \subseteq I \cap J$.
	\item If $I$ and $J$ satisfy $I+J = R$, then $IJ = I \cap J$.
	\item In general, $IJ \neq I \cap J$. Find an example of a ring $R$ and ideals $I$ and $J$ with $IJ \neq I \cap J$.
\end{enumerate}
\end{problem}

\vfill

\begin{problem}
	A prime ideal $P$ in a ring $R$ is a \emph{minimal prime} if for every prime ideal $Q$ in $R$, 
	$$Q \subseteq P \implies Q = P.$$
\vspace{-0.5em}
	Show that every prime ideal in a ring $R$ contains some minimal prime.
\end{problem}

\vfill
\vspace{1em}

\noindent
\fbox{\begin{minipage}{\textwidth}

Let $R$ be a ring, and $M$ an $R$-module. The \emph{Nagata idealization} of $(R,M)$ is the ring $R \rtimes M$ such that
\begin{itemize}
\item as a set, $R \rtimes M = R\times M$;
\item the addition is $(r,m) + (s,n)=(r+s,m+n)$;
\item the multplication is $(r,m)  (s,n)=(rs,sm+rn)$.
\end{itemize}

Then $R \rtimes M$ with the operations specified about is a ring.
\end{minipage}} 

\vspace{1em}

\vfill


\begin{problem} Consider an extension of rings $A \subseteq B \subseteq C$. In this problem, we will construct an example of such an extension such that $A \subseteq C$ is module-finite, but $A \subseteq B$ is not.\footnote{We remarked before that such examples exist, but we didn't construct one.}

\begin{enumerate}[a)]
\item Can you find such an extension with $A$ Noetherian?
\item Let $R$ be a ring that is not Noetherian, and $I$ an ideal that is not finitely generated. Show that\footnote{Note that $R$ is a subring of $R\rtimes M$ (via the inclusion $r\mapsto (r,0)$), and as an $R$-module, $R\rtimes M\cong R\oplus M$.} ${R \subseteq R \rtimes I \subseteq R \rtimes R}$, that $R \subseteq R \rtimes R$ is module-finite, but  $R \subseteq R \rtimes I$ is not.
\end{enumerate}
\end{problem}

\vfill

\begin{problem}$\,$
	\begin{enumerate}[a)]
		\item In Macaulay2, set up $A = \mathbb{Q}[s^2,st,t^2]$ as a $\mathbb{N}^2$-graded ring with the grading induced by setting $s^2, st, t^2$ as homogeneous elements of degrees
		$$\deg(s^2) = (2,0) \quad \deg(st) = (1,1) \quad \deg(t^2) = (0,2).$$
		\item The ring $R=k[t^3,t^{13},t^{42}]$ is a graded subring of $k[t]$ with the standard grading, meaning that the graded structure on $k[t]$ induces a grading on $R$. Set up $R$ (with this grading) in Macaulay2.
	\end{enumerate}
\end{problem}

\vfill

\newpage

\begin{problem}
The curve $C$ parametrized by 
$$\lbrace (t^3,t^4,t^5): t \in \mathbb{Q} \rbrace$$
in $\mathbb{A}^3_{\mathbb{Q}}$ is a variety. Use Macaulay2 to find $\mathcal{I}(C) \subseteq \mathbb{Q}[x,y,z]$. Is $C$ irreducible?
\end{problem}


\begin{problem}
	Let $X$ be the solution set of the system of equations
	$$\left\lbrace \begin{array}{l} y^{4}-2\,x\,y^{2}z+x^{2}z^{2} = 0 \\ x^{4}y^{3}-x^{5}y\,z-y^2 z^3 + x z^4 = 0 \\ x^5 y^2 - x^6 z - y^3 z^2 + x y z^{3} = 0 \\ x^9+x^3 y^3 z - 3 x^4 y z^2 + z^5 = 0 \end{array} \right.$$
	over $\mathbb{Z}/73$. Find $\mathcal{I}(X)$.
\end{problem}



\begin{problem}
Show that the functions $\mathcal{Z}$ and $\mathcal{I}$ have the following properties:
\begin{enumerate}[a)]
\item If $I=(T)$ is the ideal generated by the elements of $T$, then $\mathcal{Z}(T) = \mathcal{Z}(I)$.
\item For any field $k$, we have $\mathcal{Z}(0) = \mathbb{A}^n_k$ and $\mathcal{Z}(1) = \emptyset$. 
\item $\mathcal{I}(\emptyset)  = (1)= K[x_1, \dots, x_n]$ (the improper ideal).
\item $\mathcal{I}(\mathbb{A}^n_k) = (0)$ if and only if $k$ is infinite.
\item If $I \subseteq J\subseteq K[x_1,\ldots, x_n]$ then $\mathcal{Z}(I) \supseteq \mathcal{Z}(J)$.
\item If $S \subseteq T$ are subsets of $\mathbb{A}_k^n$ then $\mathcal{I}(S) \supseteq \mathcal{I}(T)$.
\end{enumerate}
\end{problem}

\begin{problem} In this problem, we will show that the union and intersection of varieties is a variety.
\begin{enumerate}[a)]
	\item Given two ideals $I$ and $J$ in $k[x_1, \ldots, x_d]$, $\mathcal{Z}(I) \cap \mathcal{Z}(J) = \mathcal{Z}(I + J)$.
	\item Given two ideals $I$ and $J$ in $k[x_1, \ldots, x_d]$, $\mathcal{Z}(I) \cup \mathcal{Z}(J) = \mathcal{Z}(IJ) = \mathcal{Z}(I \cap J)$.
\end{enumerate}
\end{problem}
\end{document}