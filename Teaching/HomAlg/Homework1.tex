\documentclass[11pt]{article}
\usepackage[margin=1in]{geometry}
\usepackage{amsmath,amsfonts,amssymb,amsthm,mathrsfs}
\usepackage[]{graphicx}
\usepackage{color,subfigure}
\usepackage{multicol}
\usepackage{enumerate}
\usepackage{float}
\usepackage[all]{xypic}
\usepackage[colorlinks=true,citecolor=cyan,linkcolor=magenta]{hyperref}

\newcommand{\ZZ}{\mathbb{Z}}
\DeclareMathOperator{\Hom}{Hom}

\usepackage{fancyhdr, lastpage}
\pagestyle{fancy}
\fancyfoot[C]{{\thepage} of \pageref{LastPage}}


\title{}
\date{\vspace{-0.5in}}

\makeatletter
\g@addto@macro\@floatboxreset\centering
\makeatother

\theoremstyle{definition}
\newtheorem{problem}{Problem}


\title{Problem Set 1}

\begin{document}

\thispagestyle{fancy}
\pagestyle{fancy}
\rhead{UCR $\mid$ Elo\'isa Grifo}
\lhead{Homological Algebra Spring 2021}


\begin{center}
	{\LARGE Problem Set 1}
\end{center}

\vspace{3em}

\setcounter{problem}{-1}

\begin{problem}
	Install Macaulay2.\footnote{You can only choose this problem if you did not do the corresponding problem in Commutative Algebra in the Winter.}
\end{problem}



\begin{problem}
	Let $R = \mathbb{Q}[x,y,z]/(x^2,xy)$. Consider the bounded complex
	$$C = \xymatrix@R=1mm@C=30mm{R \ar[r]^-{\begin{pmatrix} z \\ -y \\ x \end{pmatrix}} & R^3 \ar[r]^-{\begin{pmatrix} -y & -z & 0 \\ x & 0 & -z \\ 0 & x & y \end{pmatrix}
} & R^3 \ar[r]^-{\begin{pmatrix} x & y & z\end{pmatrix}
} & R \\ \text{{\tiny 3}} & \text{{\tiny 2}} & \text{{\tiny 1}} & \text{{\tiny 0}}}$$
Set $C$ up in Macaulay2 and compute its homology. For which $n$ is $\textrm{H}_n(C) = 0$?
\end{problem}



\begin{problem}
	Let $C_n = \mathbb{Z}/8$ for all $n \geqslant 0$, and $C_n = 0$ for $n<0$. Let 
	$$\xymatrix@R=2mm{C_n \ar[r]^-{d_n} & C_{n-1} \\ x \ar[r] & 4x}$$
	when $n>0$, and otherwise let $d_n\!: C_n \longrightarrow C_{n-1}$ be the zero map.
	\begin{enumerate}[a)]
		\item Show that $(C_\bullet,d_\bullet)$ is a complex.
		\item Compute its homology.
	\end{enumerate}
\end{problem}


\begin{problem}[The Five Lemma]
	Consider the following commutative diagram of $R$-modules with exact rows:
	$$\xymatrix{A' \ar[d]_-a \ar[r] & B' \ar[r] \ar[d]_-b & C' \ar[r] \ar[d]_-c & D' \ar[d]_-d \ar[r] & E' \ar[d]_-e \\
	A \ar[r] & B \ar[r] & C' \ar[r] & D' \ar[r] & E'}$$
	Show that if $a$, $b$, $d$, and $e$ are isomorphisms, then $c$ is an isomorphism.
\end{problem}


\begin{problem}
	Let $\pi$ denote the canonical projection map from $\ZZ$ to $\ZZ/2\ZZ$. Show that the chain map $f\!: F \longrightarrow G$ given by
	$$\xymatrix@R=1mm{F = & \cdots \ar[r] & 0 \ar[r] \ar[dddd]_-0 & \ZZ \ar[r]^-2 \ar[dddd]_-0 \ar[r] & \ZZ \ar[dddd]^-{\pi} \ar[r] & 0 \ar[dddd]_-0 \ar[r] & \cdots \\
	&&&&\\ 
	&&&&\\ 
	&&&&\\ 
	G = &\cdots \ar[r] & 0 \ar[r] & 0 \ar[r] & \ZZ/2\ZZ \ar[r] & 0 \ar[r] & \cdots \\
	&& \textrm{\tiny 2} & \textrm{\tiny 1} & \textrm{\tiny 0} &\textrm{\tiny -1}}$$
	is a quasi-isomorphism, but not a homotopy equivalence.
\end{problem}


\newpage

\begin{problem}$\,$
		\begin{enumerate}[a)]
		\item Show that in any category, every isomorphism is both an epi and a mono.
		\item Show that the usual inclusion $\mathbb{Z} \longrightarrow \mathbb{Q}$ is an epi in the category {\bf Ring}. This \emph{should} feel weird: it says being epi and being surjective are \emph{not} the same thing.
		\item Show that the canonical projection $\mathbb{Q} \longrightarrow \mathbb{Q}/\ZZ$ is a mono in the category of divisible abelian groups.\footnote{An abelian group $A$ is divisible if for every $a \in A$ and every positive integer $n$ there exists $b \in A$ such that $nb = a$.} Again, this is very strange: it says being monic and being injective are \emph{not} the same thing. 
		\end{enumerate}
\end{problem}


\begin{problem}
	Consider the category $R\textbf{-mod}$ of $R$-modules and $R$-module homomorphisms. Let $f$ be a homomorphism of $R$-modules. Show that $f$ is injective if and only if it is a monomorphism in $R\textbf{-mod}$, and that $f$ is surjective if and only if it is an epimorphism in $R\textbf{-mod}$.
\end{problem}


\begin{problem}
	Show that the bijection in the proof of the Yoneda Lemma is natural on both the object and the functor.
\end{problem}



\begin{problem} $\,$
	\begin{enumerate}[a)]
		\item The forgetful functor $\textbf{Grp} \longrightarrow \textbf{Set}$ is representable.
		\item Given a ring $R$, the forgetful functor $R\textbf{-mod} \longrightarrow \textbf{Set}$ is representable.
	\end{enumerate}
\end{problem}




\end{document}