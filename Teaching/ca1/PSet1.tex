\documentclass[11pt]{article}
\usepackage[margin=1in]{geometry}
\usepackage{amsmath,amsfonts,amssymb,amsthm,enumerate}
\usepackage[]{graphicx}
\usepackage{color,subfigure}
\usepackage{multicol}
\usepackage{float}
\usepackage[all]{xypic}
\usepackage[colorlinks=true,citecolor=cyan,linkcolor=magenta]{hyperref}

\usepackage{fancyhdr, lastpage}
\pagestyle{fancy}
%\fancyfoot[C]{{\thepage} of \pageref{LastPage}}


\title{}
\date{\vspace{-0.5in}}

\makeatletter
\g@addto@macro\@floatboxreset\centering
\makeatother

\theoremstyle{definition}
\newtheorem{problem}{Problem}


\title{Problem Set 1}

\begin{document}

\pagenumbering{gobble}

\thispagestyle{fancy}
\pagestyle{fancy}
\rhead{UNL $\mid$ Elo\'isa Grifo}
\lhead{Commutative Algebra 1 Fall 2022}

\


\begin{center}	
	{\LARGE Problem Set 1}
\end{center}

\vspace{3em}

\noindent
{\bf Instructions:}
For full credit, turn in 5 problems, split between a pdf and a .m2 file (one of each). 
You are welcome to work together with your classmates on all the problems, and I will be happy to give you hints or discuss the problems with you, but you should write up your solutions by yourself.
You cannot use any resources besides me, your classmates, our course notes, and the Macaulay2 documentation.

\vspace{2em}




\begin{problem} 
Install Macaulay2.%\footnote{If you don't have access to a computer, or if your computer runs only Windows, come talk to me about it.}
\end{problem}


\begin{problem}[Modules]
	Consider the domain $R = \mathbb{Q}[x,y,z,a,b,c]/(xb-ac,yc-bz,xc-az)$. Set up the following $R$-modules, making sure Macaulay2 actually sees them as modules over $R$:
	\begin{enumerate}[a)]
		\item The ideal $I = (x,y,z)$ viewed as an $R$-module.
		\item The $R$-module $N = \mathbb{Q}$.
		\item The $2$-generated $R$-module $M = Rf + Rg$, where the generators $f, g$ satisfy the relations 
		$$af-xg = 0 \quad bf - yg = 0 \quad cf - zg = 0.$$
		\item The submodule of $R^3$ generated by $(a,b,c)$ and $(x,y,z)$.
	\end{enumerate}
\end{problem}


\begin{problem}[Subalgebras]
	In Macaulay2, set up the following rings:
	\begin{enumerate}[a)]
		\item The $\mathbb{Q}$-algebra $\mathbb{Q}[xy,xu,yv,uv] \subseteq \mathbb{Q}[x,y,u,v]$.
		\item The $k$-algebra $U$, where $k = \mathbb{F}_{73}$ and
		$$U = k \begin{bmatrix} ux & uy & uz \\ vx & vy & vz \end{bmatrix} \subseteq \frac{k[u,v,x,y,z]}{(x^3+y^3+z^3)}.$$
	\end{enumerate}
\end{problem}


\end{document}