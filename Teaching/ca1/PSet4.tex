\documentclass[11pt]{article}
\usepackage[margin=1in]{geometry}
\usepackage{amsmath,amsfonts,amssymb,amsthm,enumerate}
\usepackage[]{graphicx}
\usepackage{color,subfigure}
\usepackage{multicol}
\usepackage{float}
\usepackage[all]{xypic}
\usepackage[colorlinks=true,citecolor=cyan,linkcolor=magenta]{hyperref}
\usepackage{colonequals}

\usepackage{fancyhdr, lastpage}
\pagestyle{fancy}
\fancyfoot[C]{{\thepage} of \pageref{LastPage}}


\DeclareMathOperator{\mSpec}{mSpec}
\DeclareMathOperator{\Spec}{Spec}
\DeclareMathOperator{\Ass}{Ass}
\DeclareMathOperator{\Min}{Min}
\DeclareMathOperator{\Supp}{Supp}

\newcommand{\Z}{\mathbb{Z}}
\newcommand{\m}{\mathfrak{m}}

\title{}
\date{\vspace{-0.5in}}

\makeatletter
\g@addto@macro\@floatboxreset\centering
\makeatother

\theoremstyle{definition}
\newtheorem{problem}{Problem}


\title{Homework 1}

\begin{document}

\thispagestyle{fancy}
\pagestyle{fancy}
\rhead{UNL $\mid$ Elo\'isa Grifo}
\lhead{Commutative Algebra I Fall 2022}

\vspace{3em}

\begin{center}
	{\LARGE Problem Set 4}
\end{center}

\

\noindent
{\bf Instructions:}
For full credit, turn in 5 problems in a pdf file and a .m2 file. 
You are welcome to work together with your classmates on all the problems, and I will be happy to give you hints or discuss the problems with you, but you should write up your solutions by yourself.
You cannot use any resources besides me, your classmates, our course notes, and the Macaulay2 documentation.


\vspace{2em}



\noindent
\fbox{\begin{minipage}{\textwidth}
A topological space $X$ is {\bf disconnected} if there exist disjoint, nonempty closed subsets $Y$ and $Z$ such that $X = Y \cup Z$. A topological space is {\bf connected} if no such $X$ and $Y$ exist.
\end{minipage}} 

\vspace{2em}

\noindent
\fbox{\begin{minipage}{\textwidth}
Let $R$ be a ring. An element $e \in R$ is a {\bf nontrivial idempotent} if $e^2 = e$ and $e \neq 0, 1$.
\end{minipage}}

\

\begin{problem}
Let $R$ be a ring. 
\begin{enumerate}[a)]
	\item Show that $\Spec(R)$ is connected if and only if $R$ does not contain any nontrivial idempotent.
	\item Show that if $R$ is a local ring or a domain, then $\Spec(R)$ is connected.
\end{enumerate}
\end{problem}





 
\begin{problem}
	Let $R$ be a noetherian ring and $I$ and $J$ be ideals in $R$. Show that the following are equivalent:
	\begin{enumerate}[(1)]
		\item $I \subseteq J$
		\item $I_P \subseteq J_P$ for all $P \in \Spec(R)$.
		\item $I_P \subseteq J_P$ for all $P \in \mSpec(R)$.
	\end{enumerate}
\end{problem}






\begin{problem}
	Use Macaulay2 to help you answer the following questions. Consider the ideal $I$ defining the $\mathbb{Q}$-algebra $R=\mathbb{Q}[t^3,t^{17},t^{19}]$ as a quotient of a polynomial ring in three variables.
	\begin{enumerate}[(a)]
		\item Find the minimal number of generators of $I$.
		\item Find $\Min(I)$ and $\Min(I^2)$ with Macaulay2, and give a proof that Macaulay2 is correct.
		\item Is $\Ass(I) = \Ass(I^2)$?
		\item Is $I$ a primary ideal? Give a complete answer (with a proof!) without using Macaulay2.
		\item Find a primary decomposition for $I^2$.
	\end{enumerate}
\end{problem}


\begin{problem}
	Let $R = \mathbb{Q}[x,y,z]$ and $I = (xy^2,yz)$. Find a prime filtration for the module $R/I$. You are allowed to use Macaulay2 for your computations, but your pdf submission must include a detailed explanation of the process. 
\end{problem}



\begin{problem}
	Recall that an element $x$ in a ring $R$ is regular if $xa = 0 \implies a = 0$. 
	\begin{enumerate}[a)]
		\item Show that if $x$ is a regular element, then $(x^{n+1} : x^{n}) = (x)$ for all $n \geqslant 1$.
		\item Show that if $x$ is a regular element, then $\Ass(x^n) = \Ass(x)$ for all $n \geqslant 1$.
		
		\noindent
		Hint: Consider the short exact sequence induced by the quotient map $R/(x^{n+1}) \to R/(x^n)$.
		
		\item Let $R = k[x,y,z]/(xy-z^c)$ where $k$ is a field and $c \geqslant 2$. Show that $(x^n)$ is primary for all $n \geqslant 1$. What is the radical of $(x^n)$?
		
		
		\noindent
		Hint: Show that $x$ is a regular element in $R$, and use part (b).
	\end{enumerate}	
\end{problem}


\newpage

\noindent
\fbox{\begin{minipage}{\textwidth}
Let $R = k[x_1,\ldots, x_n]$ be a polynomial ring with coefficients in a field $k$. 
We say that an ideal of $R$ is a {\bf monomial ideal} if it can be generated by monomials. 

For a polynomial $h \in R$ we say that a monomial $\mu$ is in the support of $h$ if the (unique) way to write $h$ as a sum of nonzero homogeneous polynomials has a monomial that is a scalar multiple of $\mu$. For example, for $h=x^3y+x^2-x^2+5y^3$, the support is $\{x^3y, y^3\}$ and $x^2$ is not in the support of $h$.
\end{minipage}}


\begin{problem}
Suppose $R = k[x_1,\ldots, x_n]$ is a polynomial ring with coefficients in a field $k$. 
Let $I$ be a monomial ideal of $R$
\begin{enumerate}[a)]
\item Prove that $h\in I$ if and only if each monomial in the support of $h$ is divisible by a monomial in~$I$. 
\item Prove that a monomial ideal of $R$ is prime if and only if it is generated by a subset of the variables.
\item Prove that if $f,g$ are monomials that are coprime, i.e. $\gcd(f,g)=1$, then 
$$(fg)+I=((f)+I)\cap ((g)+I).$$
\item Prove that a monomial ideal of $R$ is irreducible if and only if is of the form $(x_{i_1}^{d_1}, \ldots, x_{i_k}^{d_k})$ for some $1 \leqslant i_1< \cdots <i_k \leqslant n$ and $d_i \geqslant 1$.
%\item Use (c) to produce an irreducible decomposition for the ideal $I = (x^3y, xy^2)\subseteq k[x,y]$. Is that an irredundant primary decomposition for $I$?
%\item Use (c) to produce an irreducible decomposition for the ideal $\m^2 = (x^2,xy, y^2)\subseteq k[x,y]$. Is that an irredundant primary decomposition for $\m^2$?
\end{enumerate}
\end{problem}


\begin{problem}
Let $k$ be any field $R = k[x,y]$. Consider the following monomial ideals:
$$I = (x^3y, xy^2) \qquad \textrm{and} \qquad \m^2 = (x^2,xy, y^2)$$
\begin{enumerate}[a)]
\item Use the previous problem to produce an irreducible decomposition for $I$ and $\m^2$.
\item Are the decompositions you found irredundant primary decompositions? If not, find an irredundant primary decomposition for each ideal.
\item Find $\Min(I)$, $\Ass(I)$, $\Min(\m^2)$ and $\Ass(\m^2)$.
\end{enumerate}
\end{problem}

\end{document}