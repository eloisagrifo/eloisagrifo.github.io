\documentclass[11pt]{article}
\usepackage[margin=1in]{geometry}
\usepackage{amsmath,amsfonts,amssymb,amsthm,enumerate}
\usepackage[]{graphicx}
\usepackage{color,subfigure}
\usepackage{multicol}
\usepackage{float}
\usepackage[all]{xypic}
\usepackage{colonequals}
\usepackage{hyperref}

\usepackage{fancyhdr, lastpage}
\pagestyle{fancy}
\fancyfoot[C]{{\thepage} of \pageref{LastPage}}

\DeclareMathOperator{\Ass}{Ass}
\DeclareMathOperator{\Min}{Min}

\title{}
\date{\vspace{-0.5in}}

\makeatletter
\g@addto@macro\@floatboxreset\centering
\makeatother

\theoremstyle{definition}
\newtheorem{problem}{Problem}


\title{Problem Set 0 \\ Introductory Macaulay2 problems}


\begin{document}

\thispagestyle{fancy}
\pagestyle{fancy}
\rhead{UNL $\mid$ Elo\'isa Grifo}
\lhead{Math 918 $\mid$ Spring 2022}


\begin{center}
	{\LARGE Problem Set 1\\
	
	\vspace{0.5em}
	
	Primary Decomposition and the definition of symbolic powers}
\end{center}

\

\begin{problem}
	Let $R = \mathbb{Z}[\sqrt{-5}]$. While $6 \in R$ cannot be written as a unique product of irreducibles, we are going to show that the ideal $I = (6)$ does have a unique primary decomposition. Unfortunately, Macaulay2 cannot take primary decompositions over $\mathbb{Z}$, but this one we can do the old fashioned way.	
	\begin{enumerate}[a)]
	\item Prove that $(2)$ is a primary ideal.
	\item Prove that $(3)$ is \emph{not} a primary ideal.
	\item Prove that $(3, 1+\sqrt{-5})$ and $(3, 1-\sqrt{-5})$ are both primary.
	\item Show that $(6) = (2) \cap (3, 1+\sqrt{-5}) \cap (3, 1 -\sqrt{-5})$.
	\item Show that this primary decomposition is unique.
	\end{enumerate}
\end{problem}

\begin{problem}
	Let $I$ and $J$ be ideals in a noetherian ring $R$. Show that $I \subseteq J$ if and only if $I_P \subseteq J_P$ for all $P \in \Ass(J)$.
\end{problem}



\begin{problem}[2 points]
Let $I$, $J$, and $L$ be ideals in a noetherian ring $R$.
\begin{enumerate}[a)]
	\item There exists $n$ such that $(I : J^\infty) = (I : J^n)$.
	\item If $Q$ is a $P$-primary ideal, then
	$$(Q : J^\infty) = \left\lbrace \begin{array}{ll} Q & \textrm{if } J \nsubseteq P \\ R & \textrm{if } J \subseteq P \end{array}\right. .$$

	\item $(I \cap J : L^\infty) = (I : L^\infty) \cap (J : L^\infty)$.
	\item Given a primary decomposition $I = Q_1 \cap \cdots \cap Q_k$,
	$$(I : J^\infty) = \bigcap_{J \nsubseteq \sqrt{Q_i}} Q_i.$$
	\item If $\Ass(I) = \Min(I) = \{ P_1, \ldots, P_k \}$, then for each $i$ there exists an element $x_i \in R$ such that the $P_i$-primary component of $I$ is given by $(I : x_i^\infty)$.
	\end{enumerate}
\end{problem}




\begin{problem}
		Let $I$ be an ideal with no embedded primes in a noetherian ring $R$. Show that there exists an ideal $J$, which we can take to be principal, such that $I^{(n)} = (I^n : J^\infty)$ for all $n \geqslant 1$.
\end{problem}


\begin{problem}[Minimal primes and support]$\,$
	\begin{enumerate}[a)]
		\item Describe $\textrm{supp}(I/I^2)$, where $I = (xz)$ in $R = \mathbb{C}[x,y,z]/(xy,yz)$.		\item Find all the minimal primes of $J = (ab,bc,cd,ad)$ in $k[a,b,c,d]$ over any field $k$.
		\item Find the minimal primes of the ring $S$, where
		$$S = \mathbb{Q} \begin{bmatrix} ux & uy & uz \\ vx & vy & vz \end{bmatrix} \subseteq \frac{\mathbb{Q}[u,v,x,y,z]}{(x^3+y^3+z^3)}.$$
	\end{enumerate}
\end{problem}



\noindent
\fbox{\begin{minipage}{\textwidth}
Fix a field $k$. The {\bf $n$th Veronese ring in $d$ variables} is the subalgebra of $R = k[x_1, \ldots, x_d]$ generated by all the degree $n$ monomials in $x_1, \ldots, x_d$, which we will denote by $R^{(n)}$. For example, $k[x,y]^{(2)} = k[x^2,xy,y^2]$. There are $N \colonequals {n+d-1 \choose n}$ algebra generators for $R^{(n)}$, so we can write it as a quotient of $k[y_1, \ldots, y_N]$; more precisely, the map $\pi: k[y_1, \ldots, y_N] \to R^{(n)}$ that sends each $y_i$ to a different monomial $x_1^{a_1} \cdots x_d^{a_d}$ with $a_1 + \cdots + a_d = n$ is surjective, and if $P = \ker \pi$, $R^{(n)} \cong k[y_1, \ldots, y_N]/P$. We call this ideal $P$ the {\bf defining ideal} of $R^{(n)}$, though $P$ is only defined up to the order of $y_1, \ldots, y_N$.
\end{minipage}} 

	
	
	

\begin{problem}
Let $P$ be the defining ideal of $\mathbb{Q}[x,y]^{(3)}$ and $Q$ be the defining ideal of $\mathbb{Q}[x,y,z]^{(2)}$.
\begin{enumerate}[a)]
	\item Is $P$ a prime ideal? Is $Q$ a prime ideal? Why?
	\item Without running any packages besides those that come preloaded with Macaulay2, find $P^{(2)}$ and $Q^{(2)}$.
	\item Is $P^2 = P^{(2)}$? Is $Q^2 = Q^{(2)}$?
\end{enumerate}
\end{problem}


\end{document}