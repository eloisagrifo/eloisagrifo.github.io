\documentclass[11pt]{article}
\usepackage[margin=0.9in]{geometry}
\usepackage{amsmath,amsfonts,amssymb,amsthm}
\usepackage{enumitem}
\usepackage{multicol}
\usepackage[color,matrix,arrow]{xy}
%\usepackage[colorlinks=true,citecolor=cyan,linkcolor=magenta]{hyperref}
\usepackage{colonequals}

\usepackage{fancyhdr, lastpage}
\pagestyle{fancy}
%\fancyfoot[C]{{\thepage} of \pageref{LastPage}}



\DeclareMathOperator{\mSpec}{mSpec}
\DeclareMathOperator{\Spec}{Spec}
\DeclareMathOperator{\Ass}{Ass}
\DeclareMathOperator{\Supp}{Supp}
\DeclareMathOperator{\height}{height}
\DeclareMathOperator{\Hom}{Hom}
\DeclareMathOperator{\ann}{ann}
\DeclareMathOperator{\End}{End}
\DeclareMathOperator{\coker}{coker}
%\DeclareMathOperator{\ker}{ker}
\DeclareMathOperator{\rank}{rank}
\DeclareMathOperator{\im}{im}
\DeclareMathOperator{\M}{M}
\DeclareMathOperator{\Tor}{Tor}
\DeclareMathOperator{\id}{id}
\DeclareMathOperator{\ch}{char}
\DeclareMathOperator{\Aut}{Aut}
\DeclareMathOperator{\Perm}{Perm}
\DeclareMathOperator{\GL}{GL}
\DeclareMathOperator{\sign}{sign}
\DeclareMathOperator{\Syl}{Syl}
\def\norm{\mathrel{\unlhd}}
\newcommand{\sdp}{\rtimes}
%\DeclareMathOperator{\dim}{dim}

\DeclareMathOperator{\lcm}{lcm}

\def\ra{\rightarrow}
\newcommand{\m}{\mathfrak{m}}
\newcommand{\C}{\mathbb{C}}
\newcommand{\Q}{\mathbb{Q}}
\newcommand{\R}{\mathbb{R}}
\newcommand{\N}{\mathbb{N}}
\newcommand{\ov}[1]{\overline{#1}}

\newcommand{\Z}{\mathbb{Z}}
\newcommand{\CC}{\mathbb{C}}
\newcommand{\ZZ}{\mathbb{Z}}
\DeclareMathOperator{\Zc}{Z}
\DeclareMathOperator{\Orb}{Orb}
\DeclareMathOperator{\Stab}{Stab}
\DeclareMathOperator{\Mat}{Mat}
\DeclareMathOperator{\ev}{ev}
\newcommand{\cP}{\mathcal{P}}
\newcommand{\cB}{\mathcal{B}}


\title{}
\date{\vspace{-0.5in}}

\makeatletter
\g@addto@macro\@floatboxreset\centering
\makeatother

\theoremstyle{definition}
\newtheorem{problem}{Problem}
\newtheorem*{solution}{Solution}
\newtheorem{lemma}{Lemma}


\begin{document}

\thispagestyle{fancy}
\pagestyle{fancy}
\rhead{UNL $\mid$ Fall 2024}
\lhead{Introduction to Modern Algebra I}

\vspace{3em}

\begin{center}
	{\LARGE Problem Set 11\\}
	Due Wednesday, December 11
\end{center}

\vspace{0.5em}

\noindent
{\bf Instructions:}
You are encouraged to work together on these problems, but each student should hand in their own final draft, written in a way that indicates their individual understanding of the solutions. Never submit something for grading that you do not completely understand. You cannot use any resources besides me, your classmates, and our course notes.


I will post the .tex code for these problems for you to use if you wish to type your homework. If you prefer not to type, please  {\em write neatly}. As a matter of good proof writing style, please use complete sentences and correct grammar. You may use any result  stated or proven in class or in a homework problem, provided you reference it appropriately by either stating the result or stating its name (e.g. the definition of ring or Lagrange's Theorem). Please do not refer to theorems by their number in the course notes, as that can change.


\vspace{2em}




\begin{problem}
	Let $I = (2,x)$ in $R = \ZZ[x]$.
	
	\vspace{-0.5em}
\begin{enumerate}[label=(1.\arabic*), itemsep=0.1em]
		\item Show that $\mathfrak{m} = (2, x)$ is a maximal ideal.
		
		\item Show that $(2,x)$ is not a principal ideal.		
\end{enumerate}
\end{problem}



\begin{problem}
	Show that every finite domain must be a field.
\end{problem}


\begin{problem}
Consider the ring $R = \Z[x]$ and the ideal $I = (3, x^3 + x + 1)$.


\begin{enumerate}[label=(3.\arabic*),itemsep=-0.1em]
\vspace{-0.5em}
	\item Show that $R/I \cong (\Z/3)[x]/(x^3 + x + 1)$.
		
	\item Find, with proof, all the ideals of $R$ that contain $I$.
\end{enumerate}
\end{problem}


\begin{problem}
	Let $R$ be a commutative ring. Show that every proper ideal $I \neq R$ is contained in some maximal ideal of $R$.
\end{problem}



\begin{problem}
Let $R$ be a commutative ring with $1 \neq 0$. We say that $R$ is noetherian if it satisfies the following ascending chain condition: for any ascending chain of ideals 
$$I_1 \subseteq I_2 \subseteq I_3 \subseteq \cdots$$ 
there exists a positive integer $n$ such that $I_n=I_{n+k}$ for all positive integers $k$; that is, the ascending chain stabilizes. 
Prove that a ring $R$ is noetherian if and only if every ideal of $R$ is finitely generated.
\end{problem}



\end{document}

