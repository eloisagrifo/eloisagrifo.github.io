\documentclass[11pt]{article}
\usepackage[margin=0.9in]{geometry}
\usepackage{amsmath,amsfonts,amssymb,amsthm}
\usepackage{enumitem}
\usepackage[]{graphicx}
\usepackage{color,subfigure}
\usepackage{multicol}
\usepackage{float}
\usepackage[all]{xypic}
\usepackage[colorlinks=true,citecolor=cyan,linkcolor=magenta]{hyperref}
\usepackage{colonequals}

\usepackage{fancyhdr, lastpage}
\pagestyle{fancy}
%\fancyfoot[C]{{\thepage} of \pageref{LastPage}}



\DeclareMathOperator{\mSpec}{mSpec}
\DeclareMathOperator{\Spec}{Spec}
\DeclareMathOperator{\Ass}{Ass}
\DeclareMathOperator{\Supp}{Supp}
\DeclareMathOperator{\height}{height}
\DeclareMathOperator{\Hom}{Hom}
\DeclareMathOperator{\ann}{ann}
\DeclareMathOperator{\End}{End}
\DeclareMathOperator{\coker}{coker}
%\DeclareMathOperator{\ker}{ker}
\DeclareMathOperator{\rank}{rank}
\DeclareMathOperator{\im}{im}
\DeclareMathOperator{\M}{M}
\DeclareMathOperator{\Tor}{Tor}
\DeclareMathOperator{\id}{id}
\DeclareMathOperator{\ch}{char}
\DeclareMathOperator{\Aut}{Aut}
\DeclareMathOperator{\Perm}{Perm}
\DeclareMathOperator{\GL}{GL}
\def\norm{\mathrel{\unlhd}}
%\DeclareMathOperator{\dim}{dim}
\DeclareMathOperator{\Stab}{Stab}
\DeclareMathOperator{\Orb}{Orb}

\def\norm{\mathrel{\unlhd}}
\DeclareMathOperator{\lcm}{lcm}

\def\ra{\rightarrow}
\newcommand{\m}{\mathfrak{m}}
\newcommand{\C}{\mathbb{C}}
\newcommand{\Q}{\mathbb{Q}}
\newcommand{\R}{\mathbb{R}}
\newcommand{\N}{\mathbb{N}}
\newcommand{\ov}[1]{\overline{#1}}

\DeclareMathOperator{\Z}{Z}
\newcommand{\ZZ}{\mathbb{Z}}


\title{}
\date{\vspace{-0.5in}}

\makeatletter
\g@addto@macro\@floatboxreset\centering
\makeatother

\theoremstyle{definition}
\newtheorem{problem}{Problem}


\begin{document}

\thispagestyle{fancy}
\pagestyle{fancy}
\rhead{UNL $\mid$ Fall 2024}
\lhead{Introduction to Modern Algebra I}

\vspace{3em}

\begin{center}
	{\LARGE Problem Set 7 \\}
	Due Wednesday, November 5
\end{center}

\

\noindent
{\bf Instructions:}
You are encouraged to work together on these problems, but each student should hand in their own final draft, written in a way that indicates their individual understanding of the solutions. Never submit something for grading that you do not completely understand. You cannot use any resources besides me, your classmates, and our course notes.


I will post the .tex code for these problems for you to use if you wish to type your homework. If you prefer not to type, please  {\em write neatly}. As a matter of good proof writing style, please use complete sentences and correct grammar. You may use any result  stated or proven in class or in a homework problem, provided you reference it appropriately by either stating the result or stating its name (e.g. the definition of ring or Lagrange's Theorem). Please do not refer to theorems by their number in the course notes, as that can change.


\



\begin{problem}
	Let $p$ be prime and let $G$ be a group of order $p^m$ for some $m \geqslant 1$. Show that if $N$ is a nontrivial normal subgroup of $G$, then $N \cap Z(G) \ne \{e\}$. In fact, show that $|N \cap Z(G)| = p^j$ for some $j \geqslant 1$. 
\end{problem}



\begin{problem}
Prove the converse to Lagrange's theorem is false: find a group $G$ and a positive integer $d$ such that $d$ divides the order of $G$ but $G$ does not have any subgroups of order $d$.		
\end{problem}



\begin{problem}
Let $G$ be a group and $H$ a subgroup of $G$. Show that $N_G(H)/C_G(H)$ is isomorphic to a subgroup of the automorphism group $\Aut(H)$ of $H$.
\end{problem}



\begin{problem}
Let $G$ be a nonabelian group of order $21$. Find the number and the sizes of the conjugacy classes of $G$, with justification.
\end{problem}


\begin{problem}
	Let $G$ be a group acting on a set $S$.
	
\begin{enumerate}[label=(5.\arabic*)]	
\item Let $s, t \in S$ be elements in the same orbit. Show that there exists $g \in G$ such that
		$$\Stab_G(s) = g \, \Stab_G(t) \, g^{-1}.$$
\item Show that if the action is transitive, then the kernel of the associated permutation representation $\rho\!: G \to \Perm(S)$ is
$$\ker(\rho) = \bigcap_{g \in G} g \, \Orb_G(s) \, g^{-1}.$$
\item Show that if $G$ is finite, the action is transitive, and $S$ has at least two elements, then there is $g \in G$ which has no fixed point, meaning that $gs \neq s$ for all $s \in S$.
\end{enumerate}
\end{problem}

\end{document}