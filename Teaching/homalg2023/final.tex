\documentclass[11pt]{article}
\usepackage[margin=1in]{geometry}
\usepackage{amsmath,amsfonts,amssymb,amsthm}
\usepackage{enumerate}
\usepackage[]{graphicx}
\usepackage{color,subfigure}
\usepackage{multicol}
\usepackage{float}
\usepackage[all]{xypic}
\usepackage{hyperref}
\usepackage{colonequals}
\usepackage{mathrsfs} 

\usepackage{fancyhdr, lastpage}
\pagestyle{fancy}
%\fancyfoot[C]{{\thepage} of \pageref{LastPage}}

\setlength{\itemsep}{10em}


\DeclareMathOperator{\mSpec}{mSpec}
\DeclareMathOperator{\Spec}{Spec}
\DeclareMathOperator{\Ass}{Ass}
\DeclareMathOperator{\Supp}{Supp}
\DeclareMathOperator{\height}{height}
\DeclareMathOperator{\Hom}{Hom}
\DeclareMathOperator{\ann}{ann}
\DeclareMathOperator{\End}{End}
\DeclareMathOperator{\coker}{coker}
%\DeclareMathOperator{\ker}{ker}
\DeclareMathOperator{\rank}{rank}
\DeclareMathOperator{\im}{im}
\DeclareMathOperator{\M}{M}
\DeclareMathOperator{\Tor}{Tor}
\DeclareMathOperator{\id}{id}
\DeclareMathOperator{\ch}{char}
\DeclareMathOperator{\Aut}{Aut}
\DeclareMathOperator{\PO}{\mathbf{PO}}
\DeclareMathOperator{\Ch}{Ch}
\DeclareMathOperator{\HH}{H}
\newcommand{\Ob}{\mathrm{Ob}}
\newcommand{\Set}{\mathbf{Set}}
%\DeclareMathOperator{\dim}{dim}



\DeclareMathOperator{\cone}{cone}
\DeclareMathOperator{\Ext}{Ext}
\DeclareMathOperator{\pdim}{pdim}

\def\ra{\rightarrow}
\newcommand{\m}{\mathfrak{m}}
\newcommand{\C}{\mathbb{C}}
\newcommand{\Q}{\mathbb{Q}}
\newcommand{\Z}{\mathbb{Z}}
\newcommand{\R}{\mathbb{R}}
\newcommand{\N}{\mathbb{N}}
\newcommand{\ov}[1]{\overline{#1}}
\newcommand{\Rmod}{R\textbf{-Mod}}


\def\ov#1{\overline{#1}}


\title{}
\date{\vspace{-0.5in}}

\makeatletter
\g@addto@macro\@floatboxreset\centering
\makeatother

\theoremstyle{definition}
\newtheorem{problem}{Problem}


\begin{document}

\thispagestyle{fancy}
\pagestyle{fancy}
\rhead{UNL $\mid$ Fall 2023}
\lhead{Homological Algebra}

\vspace{2em}

\begin{center}
	{\LARGE Final Exam}
\end{center}

\


\noindent
{\bf Instructions:}
Turn in {\bf 4} of the following problems. 
You cannot use any resources besides me and our course notes.
In particular, you cannot discuss the problems with your classmates until after the due date, and you are not allowed to use the internet or any other textbooks as a resource.

\


\begin{problem}
	Show that for all finitely generated abelian groups $M$ and $N$ and all $i \geqslant 2$,
	$$\Ext^i_\Z(M,N) = 0.$$
\end{problem}

\vfill


\begin{problem}
	Let $(R, \m, k)$ be a commutative noetherian local ring, and let $M$ be a finitely generated $R$-module. Show that
	$$\beta_i(M) = \dim_k \left( \Tor_i^R(M,k) \right) = \dim_k \left( \Ext^i_R(M,k) \right).$$
	
	\noindent
	You do not need to justify why $\Tor_i^R(M,k)$ and $\Ext^i_R(M,k)$ are $k$-vector spaces.
\end{problem}


\vfill

\begin{problem}
Show that if $\pi\!: M \to N$ is a surjective map of $R$-modules with $M$ and $N$ both flat, then $\ker \pi$ is flat.
\end{problem}


\vfill


\begin{problem}
	Show that $\pdim_R M \leqslant d$ if and only if $\Ext^{d+1}_R(M,N) = 0$ for all $R$-modules $N$.
\end{problem}

\vfill

\begin{problem}
Let $(R,\m)$ be a commutative noetherian local ring, $M$ and $N$ be finitely generated $R$-modules, and $r \in \m$. Show that if $r$ is regular on $M$ and $\Ext^i_R(M/rM,N) = 0$ for $i \gg 0$, then $\Ext^i_R(M,N) = 0$ for $i \gg 0$.

\vspace{0.5em}

	\noindent
	Hint: Show that $\Ext^i_R(M,N)$ is a finitely generated $R$-module.
\end{problem}

\vfill


\begin{problem}
	Let $f\!: A \to B$ be a map of complexes. Show that $f$ is nullhomotopic if and only if $f$ factors through the canonical map $A \to \cone(\id_A)$.
\end{problem}

\vfill


\begin{problem} Let $\mathcal{A}$ be an abelian category.
	\begin{enumerate}[a)]
		\item Show that $\ker(x \xrightarrow{\,0\,} y) = 1_x$,  $\coker(x \xrightarrow{\,0\,} y) = 1_y$, and $\im(x \xrightarrow{0} y) = 0 \longrightarrow y$.
		\item Show that $f$ is a mono if and only if $fg = 0$ implies $g = 0$ for all $g$.
		\item Show that $f$ is a mono if and only if $\ker f = 0$.
	\end{enumerate}
\end{problem}



\end{document} 