\documentclass[11pt]{article}
\usepackage[margin=1in]{geometry}
\usepackage{amsmath,amsfonts,amssymb,amsthm}
\usepackage{enumerate}
\usepackage[]{graphicx}
\usepackage{color,subfigure}
\usepackage{multicol}
\usepackage{float}
\usepackage[all]{xypic}
\usepackage{hyperref}
\usepackage{colonequals}
\usepackage{mathrsfs} 

\usepackage{fancyhdr, lastpage}
\pagestyle{fancy}
%\fancyfoot[C]{{\thepage} of \pageref{LastPage}}

\setlength{\itemsep}{10em}


\DeclareMathOperator{\mSpec}{mSpec}
\DeclareMathOperator{\Spec}{Spec}
\DeclareMathOperator{\Ass}{Ass}
\DeclareMathOperator{\Supp}{Supp}
\DeclareMathOperator{\height}{height}
\DeclareMathOperator{\Hom}{Hom}
\DeclareMathOperator{\ann}{ann}
\DeclareMathOperator{\End}{End}
\DeclareMathOperator{\coker}{coker}
%\DeclareMathOperator{\ker}{ker}
\DeclareMathOperator{\rank}{rank}
\DeclareMathOperator{\im}{im}
\DeclareMathOperator{\M}{M}
\DeclareMathOperator{\Tor}{Tor}
\DeclareMathOperator{\id}{id}
\DeclareMathOperator{\ch}{char}
\DeclareMathOperator{\Aut}{Aut}
\DeclareMathOperator{\PO}{\mathbf{PO}}
\DeclareMathOperator{\Ch}{Ch}
\DeclareMathOperator{\HH}{H}
\newcommand{\Ob}{\mathrm{Ob}}
\newcommand{\Set}{\mathbf{Set}}
%\DeclareMathOperator{\dim}{dim}


\def\ra{\rightarrow}
\newcommand{\m}{\mathfrak{m}}
\newcommand{\C}{\mathbb{C}}
\newcommand{\Q}{\mathbb{Q}}
\newcommand{\Z}{\mathbb{Z}}
\newcommand{\R}{\mathbb{R}}
\newcommand{\N}{\mathbb{N}}
\newcommand{\ov}[1]{\overline{#1}}
\newcommand{\Rmod}{R\textbf{-Mod}}


\def\ov#1{\overline{#1}}


\title{}
\date{\vspace{-0.5in}}

\makeatletter
\g@addto@macro\@floatboxreset\centering
\makeatother

\theoremstyle{definition}
\newtheorem{problem}{Problem}


\begin{document}

\thispagestyle{fancy}
\pagestyle{fancy}
\rhead{UNL $\mid$ Fall 2023}
\lhead{Homological Algebra}

\vspace{2em}

\begin{center}
	{\LARGE Midterm}
\end{center}

\


\noindent
{\bf Instructions:}
Turn in {\bf 4} of the following problems. 
You cannot use any resources besides me and our course notes.
In particular, you cannot discuss the problems with your classmates until after the due date, and you are not allowed to use the internet or any other textbooks as a resource.


\





\begin{problem}
	Consider a short exact sequence in $\Ch(R)$, say
	$$\xymatrix{0 \ar[r] & A \ar[r] & B \ar[r] & C \ar[r] & 0.}$$
	Show that if any two of $A$, $B$, and $C$ are exact everywhere, then so is the third one.
\end{problem}

\vfill


\begin{problem}
	Let $C$ and $D$ be complexes of $R$-modules and let $f\!: C \to D$ be a map of complexes. 
	\begin{enumerate}[a)]
		\item Show that if $\ker(f)$ and $\coker(f)$ are both exact, then $f$ is a quasi-iso.
		\item Is the converse of a) true? Either prove it or give a counterexample.
	\end{enumerate}
\end{problem}


\vfill


\begin{problem} Let $R$ be a commutative ring.
\begin{enumerate}[a)]
	\item Let 
	$$\xymatrix{A \ar[r]^-f & B} \quad \textrm{and} \quad \xymatrix{B \ar[r]^-g & C}$$
	be two $R$-module homomorphisms such that for all $R$-modules $M$, 
	$$\xymatrix{\Hom_R(M,A) \ar[r]^-{f_*} & \Hom_R(M,B) \ar[r]^-{g_*} & \Hom_R(M,C)}$$
	is an exact complex. Show that
	$$\xymatrix{A \ar[r]^-f & B \ar[r]^-g & C}$$
	is an exact complex.
	\item Let $(F,G)$ be an adjoint pair of covariant additive functors $\Rmod \to \Rmod$, where $R$ is commutative; assume that for all $R$-modules $M$ and $N$, there is an isomorphism of $R$-modules	
	$$\Hom_R(F(M),N) \cong \Hom_R(M,G(N))$$
	which is natural on both $M$ and $N$. Show that $G$ must be a left exact functor.
\end{enumerate}
\end{problem}


\vfill

\begin{problem}
	Let $R$ be a commutative ring and let $M$ and $N$ be $R$-modules.
	
	\begin{enumerate}[a)]
		\item Show that if $M$ and $N$ are both projective, then $M \otimes_R N$ is projective.
		\item Show that if $M$ and $N$ are both flat, then $M \otimes_R N$ is flat.
	\end{enumerate}
\end{problem}



\vfill


\begin{problem}
	Let $k$ be a field. Give a proof for a) before proving b).
	\begin{enumerate}[a)]
		\item Show that all $k$-vector spaces are projective, injective, and flat.
		\item Show that every additive functor $k\textbf{-Mod} \longrightarrow k\textbf{-Mod}$ is exact.
	\end{enumerate}
\end{problem}


\vfill

\begin{problem}
	Let $(R, \m)$ be a commutative local ring and $M$ and $N$ be finitely generated $R$-modules. 
	\begin{enumerate}[a)]
	\item Using only properties of tensor products and without exhibiting an explicit iso, show that
	$$R/\m \otimes_R (M \otimes_R N) \cong (R/\m \otimes_R M) \otimes_{R/\m} (R/\m \otimes_R N).$$
	\item Show that if $M \otimes_R N = 0$, then $M = 0$ or $N = 0$.
	\end{enumerate}
\end{problem}


\end{document}