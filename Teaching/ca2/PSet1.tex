\documentclass[11pt]{article}
\usepackage[margin=0.9in]{geometry}
\usepackage{amsmath,amsfonts,amssymb,amsthm}
\usepackage{enumitem}
\usepackage[]{graphicx}
\usepackage{color,subfigure}
\usepackage{multicol}
\usepackage{float}
\usepackage[all]{xypic}
\usepackage[colorlinks=true,citecolor=cyan,linkcolor=magenta]{hyperref}
\usepackage{colonequals}

\usepackage{stmaryrd}
\usepackage{fancyhdr, lastpage}
\pagestyle{fancy}
%\fancyfoot[C]{{\thepage} of \pageref{LastPage}}



\DeclareMathOperator{\mSpec}{mSpec}
\DeclareMathOperator{\Spec}{Spec}
\DeclareMathOperator{\Ass}{Ass}
\DeclareMathOperator{\Supp}{Supp}
\DeclareMathOperator{\height}{height}
\DeclareMathOperator{\Hom}{Hom}
\DeclareMathOperator{\ann}{ann}
\DeclareMathOperator{\End}{End}
\DeclareMathOperator{\coker}{coker}
%\DeclareMathOperator{\ker}{ker}
\DeclareMathOperator{\rank}{rank}
\DeclareMathOperator{\im}{im}
\DeclareMathOperator{\Ext}{Ext}
\DeclareMathOperator{\Tor}{Tor}
%\DeclareMathOperator{\dim}{dim}

\DeclareMathOperator{\lcm}{lcm}

\def\ra{\rightarrow}
\newcommand{\m}{\mathfrak{m}}
\newcommand{\C}{\mathbb{C}}
\newcommand{\Q}{\mathbb{Q}}
\newcommand{\R}{\mathbb{R}}
\newcommand{\N}{\mathbb{N}}
\newcommand{\ov}[1]{\overline{#1}}

\DeclareMathOperator{\Z}{Z}
\newcommand{\ZZ}{\mathbb{Z}}
\DeclareMathOperator{\pdim}{pdim}
\DeclareMathOperator{\kos}{kos}


\title{}
\date{\vspace{-0.5in}}

\makeatletter
\g@addto@macro\@floatboxreset\centering
\makeatother

\theoremstyle{definition}
\newtheorem{problem}{Problem}


\begin{document}

\thispagestyle{fancy}
\pagestyle{fancy}
\rhead{UNL}
\lhead{Math 906}
\chead{Problem Set 1}

\vspace{3em}

\begin{center}
	{\LARGE Problem Set 1 \\}
	Due Wednesday, January 28, 2026
\end{center}

\

\noindent
{\bf Instructions:}
You are encouraged to work together on these problems, but each student should hand in their own final draft, written in a way that indicates their individual understanding of the solutions. Never submit something for grading that you do not completely understand. You cannot use any resources besides me, your classmates, and our course notes.


I will post the .tex code for these problems for you to use if you wish to type your homework. If you prefer not to type, please {\em write neatly}. As a matter of good proof writing style, please use complete sentences and correct grammar. You may use any result  stated or proven in class or in a homework problem, provided you reference it appropriately by either stating the result or stating its name (e.g. the definition of depth or Hilbert's Syzygy Theorem). Please do not refer to theorems by their number in the course notes, as that can change.


\vspace{2em}


\noindent
Turn in {\bf 4 problems} of your choosing. Any problem you do not turn is now a known theorem.


\begin{problem}
Let $(R, \m, k)$ be a noetherian local ring and let $M$ be a finitely generated $R$-module.
	Show that 
	\[ \beta_{i}(M) = \dim_k \left(  \Tor^R_i(M, k) \right) = \dim_k \left( \Ext_R^i(M,k) \right).\]
\end{problem}




\begin{problem}
	Let $R$ be a local/graded domain and $M$ be a finitely generated (graded) $R$-module.
	\begin{enumerate}[label=\alph*), itemsep=-0.1em]
		\item Show that if $F$ is any finite free resolution for $M$ over $R$, then
		\[\sum_{i} (-1)^i \rank (F_i) = \rank (M).\]
		In particular,
		\[\sum_{i} (-1)^i \beta_i(M) = \rank (M).\]
		\item Show that if $I$ is a nonzero proper ideal of $R$ and $M$ is an $R/I$-module with $\pdim_R(M) < \infty$, then
		\[ \sum_{i \geqslant 0} \beta_{2i}^R(M) = \sum_{i \geqslant 0}  \beta_{2i+1}^R(M) .\]
	\end{enumerate}	
\end{problem}



\begin{problem}
	Let $Q = k [ x,y,z, w ]$, $I = (xy,yz,zw)$, and $M = Q/I$. Assume $\pdim(M) = 2$.
	
	\begin{enumerate}[label=\alph*), itemsep=-0.1em]
		\item Find the betti numbers $\beta_i(M)$ without finding the minimal free resolution for $M$.
		\item Find the minimal free resolution for $M$, with proof.
		\item Find the graded betti numbers and the betti table of $M$. 
		\item Check your work with Macaulay2. Include your Macaulay2 code as a .m2 file with comments.
	\end{enumerate}
\end{problem}


\begin{problem}
	Let $R$ be a noetherian ring, $I = (x_1, \ldots, x_n)$ an ideal in $R$, and $M$ a finitely generated $R$-module. Show that if $IM = M$, then $\kos(\underline{x};M)$ is exact.
\end{problem}


\begin{problem}
Let $(R, \m, k)$ be a regular local ring and $\underline{x} = x_1, \ldots, x_n \in R$ and $\underline{y} = y_1, \ldots, y_m \in R$ be such that $(\underline{x}) = (\underline{y}) = I$.

\begin{enumerate}[label=\alph*), itemsep=-0.1em]
	\item Show that if $\underline{x}$ and $\underline{y}$ are minimal generating sets for $I$, then $\kos(\underline{x})$ and $\kos(\underline{y})$ are isomorphic complexes.
	\item Give an example showing that Koszul homology may depend on the choice of generators for $I$.
\end{enumerate}
\end{problem}



\newpage


\begin{problem}
	Let $R$ be a noetherian local ring and $M$ a finitely generated $R$-module with~$\pdim(M) =t$. Show that for all nonzero finitely generated $R$-modules $N$,
	\[ \Ext^t_R(M,N) \neq 0.\]
\end{problem}


\vspace{2em}

\noindent
\hspace{0.5em}
\fbox{
\begin{minipage}{0.95\textwidth}
\vspace{0.2em}
Let $k$ be a field and let $f_1, \ldots, f_n$ be monomials in $k[x_1, \ldots, x_d]$, minimally generating the ideal $I = (f_1, \ldots, f_n)$. For each subset $J \subseteq [n] := \{ 1, \ldots, n\}$, set
\[ f_J = \lcm(f_j \mid j \in J). \]



The {\bf Taylor resolution} of $R/I$ is the complex $(T,\partial)$ defined as follows:

\begin{itemize}
	\item In homological degree $s$, $T_s$ is the free $R$-module on basis $e_J$ with
	\[ J = \{ j_1, \ldots, j_s \} \subseteq [n]\]
	ranging over all the subsets of $[n]$ of size $|J|=s$.
	\item For each basis element $e_J$ with $|J|=s$, the differential is defined as
	\[ \partial(e_J) = \sum_{i=1}^s (-1)^{i+1} \frac{f_J}{f_{J \setminus \{ j_i\}}} e_{J \setminus \{ j_i \}}. \]
\end{itemize}


In her 1960s PhD thesis, Diana Taylor proved that this is a free resolution for $I$, which is now known as the {\bf Taylor resolution}. We will use her theorem without proof.

\vspace{0.4em}

Note: any monomial ideal has a unique minimal generating set consisting of monomials, so we can talk about the Taylor resolution of a monomial ideal $I$.
\vspace{0.2em}
\end{minipage}
}
 
 
\ 
 
\begin{problem}
Let $k$ be a field and let $f_1, \ldots, f_n$ be monomials in $k[x_1, \ldots, x_d]$, minimally generating the monomial ideal $I = (f_1, \ldots, f_n)$. 
	\begin{enumerate}[label=\alph*), itemsep=-0.1em]
		\item Find the Taylor resolution of $I = (xy,xz,yz) \subseteq k[x,y,z]$, clearly indicating each basis element.
		\item Use the Taylor resolution of $I = (xy,xz,yz)$ to find the minimal resolution for $I$.
		
		\noindent
		Note that the goal of this problem is to understand how to minimize a nonminimal free resolution.
		
		\item Let $I = (f_1, \ldots, f_n)$ be a squarefree monomial ideal. Show that the Taylor resolution of $I$ is minimal if and only if for each $i$ there exists a variable $y_i$ such that $y_i \mid f_i$ but $y_i \nmid f_j$ for all $j \neq i$.
	\end{enumerate}
\end{problem} 

 


 


\end{document}