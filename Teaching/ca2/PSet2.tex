\documentclass[11pt]{article}
\usepackage[margin=0.9in]{geometry}
\usepackage{amsmath,amsfonts,amssymb,amsthm}
\usepackage{enumitem}
\usepackage[]{graphicx}
\usepackage{color,subfigure}
\usepackage{multicol}
\usepackage{float}
\usepackage[all]{xypic}
\usepackage[colorlinks=true,citecolor=cyan,linkcolor=magenta]{hyperref}
\usepackage{colonequals}

\usepackage{stmaryrd}
\usepackage{fancyhdr, lastpage}
\pagestyle{fancy}
%\fancyfoot[C]{{\thepage} of \pageref{LastPage}}



\DeclareMathOperator{\mSpec}{mSpec}
\DeclareMathOperator{\Spec}{Spec}
\DeclareMathOperator{\Ass}{Ass}
\DeclareMathOperator{\Supp}{Supp}
\DeclareMathOperator{\height}{height}
\DeclareMathOperator{\Hom}{Hom}
\DeclareMathOperator{\ann}{ann}
\DeclareMathOperator{\End}{End}
\DeclareMathOperator{\coker}{coker}
%\DeclareMathOperator{\ker}{ker}
\DeclareMathOperator{\rank}{rank}
\DeclareMathOperator{\im}{im}
\DeclareMathOperator{\Ext}{Ext}
\DeclareMathOperator{\Tor}{Tor}
%\DeclareMathOperator{\dim}{dim}
\DeclareMathOperator{\HH}{H}

\DeclareMathOperator{\lcm}{lcm}

\def\ra{\rightarrow}
\newcommand{\m}{\mathfrak{m}}
\newcommand{\C}{\mathbb{C}}
\newcommand{\Q}{\mathbb{Q}}
\newcommand{\R}{\mathbb{R}}
\newcommand{\N}{\mathbb{N}}
\newcommand{\ov}[1]{\overline{#1}}


\newcommand{\Z}{\mathbb{Z}}
\DeclareMathOperator{\pdim}{pdim}
\DeclareMathOperator{\kos}{kos}
\DeclareMathOperator{\embdim}{embdim}


\title{}
\date{\vspace{-0.5in}}

\makeatletter
\g@addto@macro\@floatboxreset\centering
\makeatother

\theoremstyle{definition}
\newtheorem{problem}{Problem}


\begin{document}

\thispagestyle{fancy}
\pagestyle{fancy}
\rhead{UNL}
\lhead{Math 906}
\chead{Problem Set 2}

\vspace{3em}

\begin{center}
	{\LARGE Problem Set 2 \\}
	Due Wednesday, February 18, 2026
\end{center}

\

\noindent
{\bf Instructions:}
You are encouraged to work together on these problems, but each student should hand in their own final draft, written in a way that indicates their individual understanding of the solutions. Never submit something for grading that you do not completely understand. You cannot use any resources besides me, your classmates, and our course notes.


I will post the .tex code for these problems for you to use if you wish to type your homework. If you prefer not to type, please {\em write neatly}. As a matter of good proof writing style, please use complete sentences and correct grammar. You may use any result  stated or proven in class or in a homework problem, provided you reference it appropriately by either stating the result or stating its name (e.g. the definition of depth or Hilbert's Syzygy Theorem). Please do not refer to theorems by their number in the course notes, as that can change.


\vspace{2em}


\noindent
Turn in {\bf 5 problems} of your choosing. Any problem you do not turn is now a known theorem.

\

\noindent
\hspace{0.5em}
\fbox{
\begin{minipage}{0.95\textwidth}
\vspace{0.2em}
Let $R$ be a ring. The {\bf support} of a complex of $R$-modules $M$ is defined as
\[ \Supp(M) \colonequals \{ P \in \Spec(R) \mid M_P \text{ is not exact} \}.\vspace{0.4em} \]

\end{minipage}
}



\begin{problem}
	Let $R$ be a noetherian ring and $I = (x_1, \ldots, x_n)$ for some $\underline{x} = x_1, \ldots, x_n \in R$.
	\begin{enumerate}[label=\alph*), itemsep=-0.1em]
		\item Show that for all complexes $M$ of $R$-modules,
		\[ \Supp(M) = \bigcup_i \Supp(\HH_i(M)).\]
		\item Show that $\Supp(\kos(\underline{x})) = \Supp(R/I)$.
	\end{enumerate}
\end{problem}



\begin{problem}
Let $M$ be a finitely generated $R$-module.
\begin{enumerate}[label=\alph*), itemsep=-0.1em]
	\item Show that if $\underline{x}$ is a regular sequence on $M$, then $\Tor_1^R(M,R/(\underline{x})) = 0$.
	\item Show that if $\underline{x}$ is a regular sequence on $M$ and also on $R$, then $\Tor_i^R(M,R/(\underline{x})) = 0$ for all $i \geqslant 1$.
	\item Give an example of a ring $R$, a finitely generated $R$-module $M$, and a regular sequence $\underline{x}$ on $M$ such that $\Tor_i^R(M,R/(\underline{x})) \neq 0$ for some $i \geqslant 2$.
\end{enumerate}
\end{problem}

\begin{problem}
Let $(R, \m, k)$ be a noetherian local ring and let $F$ be a complex of finitely generated free $R$-modules, not necessarily bounded on either side.

\begin{enumerate}[label=\alph*), itemsep=-0.1em]
	\item Show that if $f$ is a regular element on $R$, then $F$ is exact if and only if $F \otimes_R R/(f)$ is exact.
	\item Show that if $\underline{f}$ is a regular sequence, then $F$ is exact if and only if $F \otimes_R R/(\underline{f})$ is exact.
	\item Show that if $R$ is regular, then  $F$ is exact if and only if $F \otimes_R R/\m$ is exact.
\end{enumerate}
\end{problem}


\begin{problem}
	Let $(R, \m)$ be a regular local ring and consider a nonzero $x \in \m$. 
		
	\begin{enumerate}[label=\alph*), itemsep=-0.1em]
	\item Show that if $x \notin \m^2$, then $R/(x)$ is a regular local ring of $\dim(R/(x)) = \dim(R)-1$.
	
	In class, we used a) to prove that regular local rings are domains.

	\item Show that if $R/(x)$ is regular then $x \in \m \setminus \m^2$.
	\end{enumerate}
\end{problem}


\begin{problem}
	Solve the Localization Problem for regular rings: if $(R, \m, k)$ is a regular local ring and $P$ is any prime ideal in $R$, then $R_P$ is a regular local ring.
\end{problem}



\newpage


\begin{problem}
	Let $(R, \m)$ be a noetherian local ring and $I$ be a proper ideal in $R$.
	\begin{enumerate}[label=\alph*), itemsep=-0.1em]
		\item Show that there is a short exact sequence
		$$\xymatrix{0 \ar[r] & \frac{I+\m^2}{\m^2} \ar[r] & \m/\m^2 \ar[r] & \frac{\m}{\m^2+I} \ar[r] & 0}$$
		and conclude that $\dim_k \left( \frac{I+\m^2}{\m^2}\right) = \embdim(R) - \embdim(R/I)$.
		\item Show that if $R$ and $R/I$ are both regular rings, then there exists a minimal set of generators $x_1, \ldots, x_d$ for $\m$ such that $I = (x_1, \ldots, x_n)$ for some $n$. In particular, $I$ must be generated by a regular sequence.
		\item If we only assume that $R/I$ is a regular ring, must $I$ be generated by a regular sequence?
	\end{enumerate}
\end{problem}



\begin{problem}
	Let $R = \Z[\sqrt{-5}] \cong \Z[x]/(x^2+5)$. Show that $R$ is a regular ring of dimension $1$, but not all maximal ideals are principal.
	
	\noindent
	Hint: show that any maximal ideal contains a prime $p \in \Z$, and consider the cases when $p-5$ is or is not a square in $\Z/(p)$ separately.
\end{problem}





\begin{problem}
	Let $R$ be a noetherian ring and consider a regular element $f \in R$. 
Show that if $R_f$ and $R/(f)$ are both regular rings, then $R$ is a regular ring.
\end{problem}


\vspace{2em}

\noindent
\hspace{0.5em}
\fbox{
\begin{minipage}{0.95\textwidth}
\vspace{0.2em}
A noetherian local ring $R$ is a complete intersection if $R \cong Q/(\underline{x})$, where $Q$ is a regular local ring and $\underline{x}$ is a regular sequence on $Q$.

\end{minipage}
}




\begin{problem}
Let $(R, \m)$ be a noetherian local ring. Show that if $R$ is a complete intersection, then so is $R_P$ for all prime ideals $P$.
\end{problem}

 


\end{document}